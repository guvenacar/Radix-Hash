% Radix-Hash Article - LaTeX Version
\documentclass\[12pt]{article}
\usepackage\[utf8]{inputenc}
\usepackage{amsmath, amssymb, amsfonts}
\usepackage{graphicx}
\usepackage{hyperref}
\usepackage{booktabs}
\usepackage{geometry}
\usepackage{algorithm}
\usepackage{algpseudocode}
\geometry{margin=1in}

\title{Radix-Hash: Problem-Independent Cryptographic Hash Algorithm and Post-Quantum Security Analysis}
\author{Güven ACAR\Institution\Email\ORCID}
\date{}

\begin{document}

\maketitle

\begin{abstract}
This paper presents the Radix-Hash algorithm, a novel approach distinct from classical cryptographic hash algorithms. Unlike conventional hash functions based on specific mathematical problems, Radix-Hash employs a purely mathematical chaos-based security model without relying on any solvable mathematical problem. Producing a 772-bit output, the algorithm uses base-3 conversion and modular arithmetic to create a one-way hash function. Passing NIST statistical tests with 99% success, Radix-Hash offers an alternative security paradigm in the post-quantum era. Performance analyses confirm the proof-of-concept success and highlight future optimization potential.

\textbf{Keywords:} Cryptographic hash, post-quantum security, base-3 conversion, mathematical chaos, NIST tests
\end{abstract}

\section{Introduction}
\subsection{Limitations of Existing Cryptographic Paradigms}
Hash algorithms typically rely on specific mathematical problems:
\begin{itemize}
\item \textbf{RSA:} prime factorization problem
\item \textbf{Diffie-Hellman and ElGamal:} discrete logarithm problem
\item \textbf{Quantum-resistant algorithms:} often lattice-based problems
\end{itemize}
This paradigm assumes the attacker cannot solve these problems efficiently; however, if such problems are solved in the future, algorithm security is compromised.

\subsection{Philosophical Approach of Radix-Hash}
Radix-Hash differs philosophically:
\begin{itemize}
\item Operates without assuming any mathematical problem
\item Provides a one-way transformation without leaving a solvable problem for the attacker
\item Executes in a few simple steps compared to iterative SHA-family operations
\item Achieves 99% success in NIST statistical tests even without additional bit-scrambling layers
\end{itemize}

\subsection{Post-Quantum Cryptography Context}
With the advent of quantum computing:
\begin{itemize}
\item \textbf{Shor's algorithm} threatens RSA and elliptic curve cryptography
\item \textbf{Grover's algorithm} reduces symmetric security by half
\item Existing post-quantum candidates (CRYSTALS-Kyber, CRYSTALS-Dilithium) still rely on specific mathematical problems
\end{itemize}
Radix-Hash, not assuming a solvable problem, is naturally quantum-resistant.

\section{Mathematical Definition of Radix-Hash}
Radix-Hash works on 772-bit blocks and does not rely on classical mathematical problems. The 772-bit length is chosen since \$3^{486} \approx 770\$ bits, close to the target length.

\subsection{Algorithm Components}
\subsubsection{Input Processing and Block Formation}
\begin{enumerate}
\item \textbf{Input to Bit Array:} UTF-8 text converted to bits
\item \textbf{Padding:} Bits split into 772-bit blocks, padded with 0s if necessary
\end{enumerate}

\subsubsection{Scrambling Layer}
\begin{enumerate}
\item Split each block: \$A|B\$
\item XOR: \$X = A \oplus B\$
\item Reverse and NOT: \$X\_{not-rev} = reverse(NOT(X))\$
\item Interleave XOR and NOT-reverse results dynamically to increase entropy
\end{enumerate}

\subsubsection{Core Transformation}
\begin{enumerate}
\item Base-3 conversion and modular exponentiation:
$     bits\_to\_base3\_int(b_1...b_n) = \sum_{i=1}^n d_i \cdot 3^{n-i},\quad d_i \in \{1,2\}
    $
Mapping: '0' \$\to\$ 1, '1' \$\to\$ 2
\item For each hexadecimal character, map 16-base to decimal, add 2, and apply modular exponentiation with \$M = 3^{486}\$
\end{enumerate}

\subsubsection{Output Formation}
All blocks XORed and normalized to 772 bits, producing a one-way, hard-to-invert hash.

\subsection{Pseudocode}
\begin{algorithm}
\caption{Radix-Hash}
\begin{algorithmic}\[1]
\Procedure{RADIX-HASH}{input\_text}
\State bits = UTF8\_TO\_BITS(input\_text)
\State bits = PAD\_TO\_772\_MULTIPLE(bits)
\State blocks = SPLIT\_INTO\_772\_BIT\_BLOCKS(bits)
\State final\_hash = 0
\For{block in blocks}
\State normalized = NORMALIZE\_TO\_772\_BITS(block)
\State scrambled = XOR\_NOT\_REVERSE\_DYNAMIC\_COUNT(normalized)
\State base3\_int = BITS\_TO\_BASE3\_INT(scrambled)
\State hex\_string = TO\_HEX(base3\_int)
\State hash\_result = MODULAR\_CHAOS\_FUNCTION(hex\_string)
\State final\_hash = final\_hash \textbf{XOR} hash\_result
\EndFor
\State \textbf{return} NORMALIZE\_TO\_772\_BITS(final\_hash)
\EndProcedure
\end{algorithmic}
\end{algorithm}

\section{Security Analysis}
\subsection{Problem-Independent Security Model}
Traditional security depends on hard math problems; Radix-Hash relies solely on chaotic mathematical transformations.

\subsection{Quantum Resistance}
\subsubsection{Against Shor's Algorithm} Radix-Hash avoids prime factorization and discrete logarithms.
\subsubsection{Against Grover's Algorithm} 772-bit output ensures \$2^{386}\$ quantum attempts are still infeasible.

\subsection{Cryptanalysis Resistance}
\begin{itemize}
\item Preimage Resistance: base-3 conversion + modular chaos
\item Collision Resistance: 772-bit output + chaotic mapping
\item Avalanche Effect: small input changes yield large output differences
\end{itemize}

\section{NIST Statistical Test Results}
Radix-Hash evaluated with NIST SP 800-22 tests on 100 sequences of 1M bits each. Success rate: 99%+.

\begin{table}\[h]
\centering
\begin{tabular}{lcc}
\toprule
Test & Pass Rate & P-Value \\
\midrule
Frequency & 99/100 & 0.924076 \\
Block Frequency & 99/100 & 0.494392 \\
Cumulative Sums & 100/100 & 0.946308 \\
Runs & 100/100 & 0.236810 \\
FFT & 100/100 & 0.637119 \\
Rank & 100/100 & 0.616305 \\
Universal & 98/100 & 0.401199 \\
Approximate Entropy & 98/100 & 0.015598 \\
Serial & 99/100 & 0.699313 \\
Linear Complexity & 99/100 & 0.366918 \\
\bottomrule
\end{tabular}
\caption{Summary of NIST Test Results}
\end{table}

\section{Performance Analysis and Comparison}
\subsection{Test Environment} 12-core CPU, 15.5GB RAM, Python 3.11.2, Linux.

\subsection{Performance Metrics}
\begin{table}\[h]
\centering
\begin{tabular}{lcccc}
\toprule
Algorithm & Small Input (ms) & Large Input (ms) & Memory (KB) & Throughput (MB/s) \\
\midrule
Radix-Hash & 3.134 & 1470.837 & 41.7-4212.3 & 0.04 \\
SHA-256 & 0.008 & 0.110 & 41.4-64.1 & 494.77 \\
SHA3-256 & 0.006 & 0.162 & 41.4-64.1 & 350.66 \\
BLAKE2b & 0.084 & 0.084 & 41.4-64.5 & 665.47 \\
\bottomrule
\end{tabular}
\caption{Performance Comparison}
\end{table}

\section{Related Work and Comparison}
Traditional SHA families, post-quantum hash approaches (SPHINCS+, XMSS/LMS), and Radix-Hash's unique contributions are discussed.

\section{Conclusion and Future Work}
Radix-Hash introduces a problem-independent hash paradigm with post-quantum resistance, verified by NIST tests and with future optimization potential.

\section\*{References}
\begin{enumerate}
\item NIST SP 800-22, 2010.
\item Shor, P. W., SIAM Journal on Computing, 1997.
\item Grover, L. K., ACM STOC, 1996.
\item NIST Post-Quantum Cryptography Standardization, 2022.
\item Bernstein, D. J., et al., SPHINCS+, 2019.
\item Keccak Team, The Keccak SHA-3 submission, 2011.
\item Aumasson, J. P., et al., BLAKE2, 2013.
\end{enumerate}

\end{document}
